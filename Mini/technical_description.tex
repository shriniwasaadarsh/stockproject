%%  Technical Chapters
%%%%%%% Note : The below mentioned chapters, sections and subsections are given in general.
%%% Only appropriate and necessary chapters, sections and subsections are to be included
%\chapter{Title of the Chapter}
%This chapter illustrates, how to refer Figure, Equations, Table and References.
\chapter{System Overview}
In this chapter an overview of the system architecture used in the project is discussed.It typically highlights key components, functionalities, significance and potential impact of the system in addressing specific challenges or fulfilling particular needs within its target domain.
\section{System Architecture}
\vspace{0.5cm}
\begin{figure} [h]
    \centering
    \includegraphics[scale=1]{basicsysarchi1.png} 
    \caption{{Basic System Architecture }}
    
\end{figure}

The diagram 3.1 depicts a simplified view of a crowdfunding platform that leverages blockchain technology. The process starts with a request for funding being created (Create Request). This request is likely to include details about the project or campaign, along with a funding target. Potential investors can then view a summary of the request (Get Summary). If interested, they can contribute funds to the project (Contribute). These transactions are recorded on the Ethereum blockchain network (Ethereum Network).

The Ethereum blockchain network uses a system of blocks (Block 1, Block 2, Block n) to securely store information.  Transactions are bundled together into these blocks, and each block is cryptographically linked to the one before it. This creates a secure and tamper-proof record of all transactions. Once the crowdfunding period concludes, investors vote to approve or reject the funding request (Vote). If the request is approved, the funds are released to the campaign creator (Finalize Request).

Blockchain technology offers several potential advantages for crowdfunding platforms.  By using a blockchain, all transactions are recorded on a public ledger, which can increase transparency and reduce the risk of fraud. Additionally,  blockchain can automate some of the processes involved in crowdfunding, such as the release of funds upon successful completion of a campaign.


\vspace{0.5cm}
\begin{figure} [h]
    \centering
    \includegraphics[scale=1]{flowinether.png} 
    \caption{{Flow of ether in proposed blockchain model }}
    
\end{figure}

\textbf{a.}The campaign manager creates a campaign, providing campaign details and optionally uploading a proposal in PDF format. The proposal is stored in a decentralized file sharing system called IPFS.

\textbf{b.} The campaign is listed on a campaign page, and interested contributors can contribute funds to it.

\textbf{c.}The campaign manager can create an expense request, listing the items required for the campaign.

\textbf{d.}Contributors are notified of new expense requests.

\textbf{e.} Contributors review the proposed items and can vote to approve them. If the majority agrees, a smart contract releases funds to vendors.

\textbf{f.}Vendors deliver the approved items to the campaign manager.

\textbf{g.}All transactions, including contributions, expense requests, and fund transfers, are recorded on the blockchain and can be viewed by all users on the Ethers can website. This transparency ensures trust and accountability in the crowdfunding process.\\

\textbf{Smart Contract Architecture:} 
-Created a primary contract to manage the crowdfunding process. \\
-Each project have its own instance of a sub-contract that handles the project-specific details.\\ 

\textbf{Project Creation:} 
-Users can create projects by deploying a new instance of the project-specific contract. \\
-Project creator specifies the funding goal, duration, and other relevant details. \\
-A unique project ID is assigned to each project.  \\

\textbf{Funding Mechanism:} 
-Users can contribute funds to projects using Ether (the cryptocurrency of the Ethereum blockchain). \\
-Funds are transferred directly to the project's contract address -Contributions are recorded in the contract's state variables.  \\

\textbf{Project Status: }
-The contract maintains the current status of each project (e.g., open, funded, expired). \\
-The status changes based on the project's funding progress and duration. \\

\textbf{Time-Based Events: }
-Use block timestamps to track the start and end times of each project. \\
-Projects will automatically close after their specified duration, regardless of reaching the funding goal.  \\

\item \textbf{Funding Goal and Refunds: }
-If a project reaches its funding goal, the funds are released to the project creator. \\
-If the funding goal is not reached within the specified duration, contributors can claim a refund. \\
-Refunds are initiated by contributors, and the contract transfers their contributed funds back to them. \\

\vspace{0.5cm}
\begin{figure} [h]
    \centering
    \includegraphics[scale=0.8]{workingmodel.png} 
    \caption{{Working model }}
    
\end{figure}
\\
 Fig 3.3  showcases a crowdfunding process powered by blockchain technology, specifically Ethereum.  It starts with a Campaign Creator outlining a funding request, detailing their project and target amount. Potential investors can then access a Summary of the request.  If interested, they can contribute funds directly to the project.  These transactions are secured on the Ethereum blockchain network, represented by Block 1, Block 2,...,Block n.  These blocks act as digital ledgers, where transactions are bundled and cryptographically linked together using "Previous Hash" references. This creates an unalterable record of every contribution.  Following the crowdfunding period, investors collectively vote to Approve or Reject the request. If approved, the platform automatically releases the funds to the Campaign Creator using smart contracts (not shown in the diagram). This automation and transparency offered by blockchain holds the potential to revolutionize crowdfunding by minimizing fraud and streamlining processes.\\

 
\\

\begin{figure} [h]
    \centering
    \includegraphics[scale=0.8]{blkchainnewq2.png} 
    \caption{{Blockchain}}
    
\end{figure}

The diagram 3.4 illustrates a simplified overview of how a blockchain-based crowdfunding platform might function. Data blocks, labeled Block 1, Block 2,...,Block n represent transactions recorded on a public ledger.  Transactions could include the creation of a funding request, contributions from investors, and approvals or rejections made by voters. Arrows depict the flow of data as it progresses through the different stages of the crowdfunding process. Text annotations, such as "Previous Hash" reference the cryptographic chaining process that secures the blockchain.\\

\section{Key Development Components}


\textbf{1.	User Authentication and Authorization:} \\
•	Implement user authentication mechanisms to secure user accounts and prevent unauthorized access. \\
•	Use cryptographic techniques such as public-private key pairs for user authentication. \\
•	Define role-based access control (RBAC) to manage permissions for project creators, backers, and administrators. \\
 
\textbf{2.	Smart Contract Development:} \\
•	Develop custom smart contracts to govern the crowdfunding process, including fundraising, fund distribution, and reward fulfillment. \\
•	Define functions for initiating fundraising campaigns, accepting contributions, releasing funds to project creators, and refunding contributions if the funding goal is not met. \\
•	Implement event logging in smart contracts to emit events for key actions (e.g., contribution received, funding goal reached) for transparency and monitoring.\\ 
\textbf{3.	Transparency and Accountability:} \\
•	Display real-time campaign metrics, including funding progress, contributions received, and remaining time, to foster transparency and encourage engagement. \\
•	Enable backers to track the allocation of funds and monitor project updates and milestones throughout the crowdfunding process. \\
 
\textbf{4.	Proof of work }(abbreviated to PoW) is a consensus protocol used widely by many cryptocurrencies. This process is known as mining and the node on the network is called as miners. The Proof of Work is a mathematical problem one that requires considerable work to achieve the solution. The only way to solve the problem is through the node in the network have to run the process based on trial-and-error basis. A miner will continue testing different unique values until a suitable hash is produced. The miner who manages to solve will add next block, adding the block to the chain and validates all the transactions within it, and receiving the reward associated with the block.  \\
 
\textbf{5.	Consensus protocol} is the most important one in the blockchain technology. The Blockchain consensus protocol is what which keeps the blocks on all the node to synchronize with each other. The term 'Consensus' means that the nodes have to agree with the same state of the blockchain. Consensus protocol allows blockchain to be updated every minute (depends on the network) and ensures that every block in the chain is true. The aim of the consensus protocol is to guarantee a single chain is used and followed by all the nodes. 
 \\


\section{Requirements}
\subsection{Functional Requirements}
\begin{itemize}
    \item \textbf{User Registration and Authentication:}\\
- Users should be able to register accounts securely.\\
- The platform must support authentication mechanisms to verify user identities.

\item \textbf{Project Creation and Management:}\\
- Project creators should be able to create crowdfunding campaigns.\\
- They should be able to provide details about their projects, including descriptions, funding goals, timelines, and rewards for backers.\\
- Project creators must have the ability to edit, update, and manage their campaigns.
\item \textbf{Investment and Funding Mechanism:}\\
- Backers should be able to browse and discover projects they wish to support.\\
- The platform should facilitate the process of making contributions using various payment methods, including cryptocurrencies.\\
- Smart contracts or other mechanisms should ensure that funds are securely held in escrow until funding goals are met or predefined conditions are satisfied.
\item \textbf{Transaction Processing and Confirmation:}\\
- The platform must process transactions securely and transparently on the blockchain.\\
- Users should receive confirmation of their transactions and contributions in real-time.
\item \textbf{Smart Contract Execution:}\\
- Smart contracts should automatically execute predefined rules and conditions, such as releasing funds to project creators when funding goals are met.\\
- The platform should support the creation and deployment of custom smart contracts tailored to each crowdfunding campaign.
\item \textbf{Transparency and Accountability:}\\
- The platform should provide transparent visibility into project details, funding progress, and transaction history.\\
- Users should be able to track how funds are being utilized by project creators.\\
- Audit trails and immutable records on the blockchain should ensure accountability and trustworthiness.
\item \textbf{Communication and Engagement:}\\
- The platform should facilitate communication and interaction between project creators and backers.\\
- Users should be able to ask questions, provide feedback, and engage with project updates.\\
- Notification mechanisms should keep users informed about campaign progress and milestones.
\item \textbf{Project Completion and Reward Fulfillment:}\\
- Once funding goals are met, the platform should facilitate the transfer of funds to project creators.\\
- Project creators should be responsible for fulfilling rewards or delivering products/services to backers as promised.
\item \textbf{Dispute Resolution:}\\
- The platform should have mechanisms in place to handle disputes between project creators and backers.\\
- Smart contracts or arbitration processes may be utilized to resolve disputes fairly and transparently.
\item \textbf{Regulatory Compliance:}\\
- The platform should adhere to relevant regulatory requirements and compliance standards, especially concerning investor protection, anti-money laundering (AML), and know your customer (KYC) regulations.
\end{itemize}

\subsection{Non-Functional Requirements}
\begin{itemize}
\item \textbf{Security:}\\
- The platform must implement robust security measures to protect user data, transactions, and funds from unauthorized access, fraud, and cyber-attacks. \\
- It should use encryption, authentication mechanisms, and access controls to safeguard sensitive information.\\
- Compliance with industry standards and regulations related to data security and privacy (e.g., GDPR, PCI DSS) should be ensured.

\item \textbf{Scalability:}\\
- The platform should be designed to handle increasing user traffic, transaction volumes, and data storage requirements as it grows.\\
- Scalability should be achieved through horizontal and vertical scaling of hardware resources, as well as efficient software architecture and optimization techniques.

\item \textbf{Performance:}\\
- The platform must be responsive and provide low-latency transaction processing to ensure a seamless user experience.\\
- It should be able to handle a large number of concurrent users and transactions without significant degradation in performance.
\item \textbf{Reliability and Availability:}\\
- The platform should be highly reliable, with minimal downtime and service interruptions.\\
- Redundancy, failover mechanisms, and backup systems should be in place to ensure continuous availability and data integrity.
\item \textbf{Interoperability:}\\
- The platform should be interoperable with external systems, such as payment gateways, blockchain networks, and regulatory compliance tools.\\
- It should support standard protocols and APIs for seamless integration with third-party services and applications.
\item \textbf{Regulatory Compliance:}\\
- The platform must comply with relevant legal and regulatory requirements governing crowdfunding, securities, and financial transactions.\\
- It should support compliance features such as KYC (Know Your Customer) verification, AML (Anti-Money Laundering) checks, and reporting functionalities.
\item \textbf{Auditability and Transparency:}\\
- The platform should maintain comprehensive audit trails and logs of user activities, transactions, and system events.\\
- It should provide transparent visibility into the crowdfunding process, including fund utilization, project updates, and transaction history.
\item \textbf{Usability and Accessibility:}\\
- The platform should be intuitive and user-friendly, with clear navigation, informative interfaces, and accessible features.\\
- It should support multiple languages, assistive technologies, and accessibility standards to accommodate diverse user needs.
\item \textbf{Compliance with Blockchain Principles:}\\
- The platform should adhere to core principles of blockchain technology, such as decentralization, immutability, and transparency.\\
- It should leverage blockchain capabilities to enhance trust, security, and accountability in the crowdfunding process.
\item \textbf{Performance Metrics and Monitoring:}\\
- The platform should track and monitor key performance indicators (KPIs) such as transaction throughput, response times, and user engagement metrics.\\
- Monitoring tools should be in place to detect and resolve performance issues proactively.
\end{itemize}

\subsection{Technology Stack}
\begin{itemize}
    \item \textbf{Blockchain Platform:}\\
- Choose a suitable blockchain platform for building the core infrastructure of the crowdfunding platform. \\
- Ethereum: Known for its smart contract functionality and wide adoption in the decentralized finance (DeFi) space.\\
- Hyperledger Fabric: Ideal for enterprise-grade applications with permissioned blockchain networks.\\
- Binance Smart Chain: Offers compatibility with Ethereum-based smart contracts while providing lower transaction fees.\\
- Smart Contracts: Develop smart contracts to automate crowdfunding processes, enforce funding rules, and manage transactions securely on the blockchain.\\
\textbf{Use programming languages such as:}\\
- Solidity: Ethereum's native language for writing smart contracts.\\
- Vyper: An alternative language for Ethereum smart contracts, known for its simplicity and security.

\item \textbf{Frontend Development:}\\
- HTML, CSS, JavaScript: Standard web development technologies for building user interfaces and frontend components.
React.js, Angular, Vue.js: Frontend frameworks for creating dynamic and interactive web applications.
\item \textbf{Backend Development:}\\
- Node.js: A JavaScript runtime environment for building scalable and efficient server-side applications.\\
- Express.js: A minimal and flexible Node.js web application framework for building APIs and handling HTTP requests.
\item \textbf{Database Management:}\\
- MongoDB, PostgreSQL: NoSQL and relational databases commonly used for storing user data, transaction records, and project details.\\
- Redis: In-memory data store for caching frequently accessed data and improving performance.\\
- Web3.js or Ethers.js: JavaScript libraries for interacting with the Ethereum blockchain and smart contracts from frontend applications. These libraries facilitate communication with the blockchain network, sending transactions, and retrieving data.
\item \textbf{Authentication and Authorization:}\\
- OpenID Connect: Standard protocols for user authentication and authorization, enabling secure access to platform resources.\\
- JWT (JSON Web Tokens): A compact and self-contained token format for securely transmitting authentication information between parties.
\item \textbf{Payment Gateways and Wallet Integration:}\\
- Integration with cryptocurrency wallets (e.g., MetaMask, Trust Wallet) for enabling users to contribute funds using cryptocurrencies.\\
- Payment gateway integration for supporting fiat currency payments (e.g., credit cards, bank transfers).
\item \textbf{DevOps and Deployment:}\\
- Docker: Containerization tool for packaging and deploying applications in a consistent and portable manner.\\
- Kubernetes: Container orchestration platform for automating deployment, scaling, and management of containerized applications.\\
- Continuous Integration/Continuous Deployment (CI/CD) pipelines for automating the build, test, and deployment process.
\item \textbf{Monitoring and Analytics:}\\
- Prometheus, Grafana: Monitoring tools for collecting metrics, visualizing performance data, and diagnosing issues in real-time.\\
- Google Analytics, Mixpanel: Analytics platforms for tracking user behavior, engagement metrics, and campaign performance.
\end{itemize}

\begin{figure} [h]
    \centering
    \includegraphics[scale=0.8]{metamaskwallet.png} 
    \caption{{MetaMask Wallet showing balance of SepoliaETH }}
    
\end{figure}

\textbf{Integration of MetaMask with Our dApp : }
 
   \textbf{MetaMask Wallet:} Enables authentication and transaction for Ethereum blockchain. \\
   \textbf{ React.js Front-End:} Interactive user interface to connect with MetaMask  initiate transactions. \\
   \textbf{ Solidity Back-End:} Smart contracts handle business logic and communicate with the Ethereum. \\
   \textbf{ Data Flow: }System sends and receives data to ensure transaction integrity and user notification. \\
   \textbf{ Final Goal: }Prepare for public domain launch, proving functionality beyond a proof of concept. \\

\begin{figure} [h]
    \centering
    \includegraphics[scale=0.65]{rpc.jpg} 
    \caption{{RPC (Remote Procedure Call) service for Ethereum,providing endpoints for                 
                    blockchain interaction }}
    
\end{figure}

In image 3.6,it depicts a simplified overview of a blockchain-based crowdfunding platform.  Here's how it works:  First, a Campaign Creator initiates the process by outlining a funding request. Investors can then view details and contribute funds to the project. These transactions are recorded on the Ethereum blockchain network, represented by blocks (Block 1, Block 2, Block n).  Blockchain technology ensures security by linking these blocks together cryptographically using "Previous Hash" references, making any alterations easily detectable. \\
Once the crowdfunding period ends, investors vote to Approve or Reject the request. If approved, the funds are automatically released to the campaign creator, potentially through smart contracts (not shown in the diagram). This automation and transparency offered by blockchain technology could significantly improve crowdfunding by minimizing fraud and streamlining processes.

\section{Summary}

This chapter provides a comprehensive summary of the blockchain-integrated crowdfunding platform's requirements and technology stacks. It outlines the essential functionalities required for the platform. Additionally, the section highlights the underlying technology stacks utilized, such as blockchain, smart contracts, and web development frameworks. 
In the next chapter, the implementation of the project is elaborated upon and discussed in detail.







%% new chapter

\chapter{Implementation}
\vspace{1cm}

To implement a blockchain-integrated crowdfunding platform using Web3.js, Solidity smart contracts, ThirdWeb CLI, and MetaMask, several steps need to be followed.\\

\textbf{Project Setup:}
Set up a new project directory for your crowdfunding platform.
Initialize a new Node.js project using npm init.
Install necessary dependencies such as Web3.js and ThirdWeb CLI.\\
\\
\textbf{Smart Contract Development:}
Write Solidity smart contracts to manage crowdfunding campaigns.
Define contract functions for creating campaigns, contributing funds, and withdrawing funds.
Implement logic for handling campaign milestones, rewards distribution, and fund management.
Compile the Solidity contracts using the Solidity compiler.\\
\\
\textbf{Frontend Development:}
Create frontend components using HTML, CSS, and JavaScript frameworks like React.js or Vue.js.
Integrate Web3.js to interact with Ethereum blockchain and smart contracts.
Set up MetaMask for user authentication and wallet integration.
Design user interfaces for browsing campaigns, contributing funds, and tracking campaign progress.
Implement event listeners to detect Ethereum network changes and update UI accordingly.\\
\\
\textbf{Backend Development:}
Set up a Node.js server using Express.js or another framework.
Create API endpoints for interacting with the blockchain and smart contracts.
Implement server-side logic for handling user authentication, data validation, and business logic.
Integrate Web3.js on the server-side to interact with Ethereum blockchain if necessary.\\
\\
\textbf{Smart Contract Deployment:}
Deploy the compiled Solidity smart contracts to the Ethereum blockchain using tools like Remix IDE or Truffle.
Record the contract addresses and ABI (Application Binary Interface) for frontend integration.
Integration with MetaMask:
Guide users through the process of installing and configuring MetaMask for interacting with the platform.
Implement MetaMask integration in the frontend to detect user accounts and handle transactions securely.\\
\\
\textbf{Testing:}
Write unit tests for Solidity smart contracts using tools like Truffle or Hardhat.
Test frontend components for functionality, responsiveness, and compatibility with different browsers.
Conduct end-to-end testing to ensure seamless interaction between frontend, backend, and smart contracts.\\
\\
\textbf{Process Payment Between Accounts:}
Users must have MetaMask installed and configured to interact with the Ethereum blockchain.
On the platform's frontend, users can initiate a payment by selecting the recipient's Ethereum address and specifying the amount of Ether to transfer.
The platform uses Web3.js to create and sign a transaction with MetaMask, transferring the specified amount of Ether from the sender's account to the recipient's account.
Once the transaction is signed and submitted to the Ethereum network, users can track its status and view the transaction details on the platform's interface.\\
\\
\textbf{Creating Campaigns:}
Registered users can create crowdfunding campaigns by providing details such as campaign title, description, funding goal, and duration.
The platform generates a unique Ethereum smart contract for each campaign, containing logic for managing contributions, milestones, and fund distribution.
Campaign creators must fund the initial deployment of the smart contract by transferring Ether to cover gas fees and contract initialization costs.
Upon successful deployment, the campaign becomes active, and users can start contributing funds to support the project.\\
\\
\textbf{Contributing Ether to Campaigns:}
Users can browse through active campaigns on the platform and select the ones they wish to support.
Upon selecting a campaign, users specify the amount of Ether they want to contribute and confirm the transaction using MetaMask.
The platform records the contribution by invoking the corresponding function on the campaign's smart contract, updating the total funds raised and contributor list.
Contributors receive tokens or other rewards based on the campaign's reward structure, as defined in the smart contract.\\
\\
\textbf{Deployment:}
Deploy the frontend application to a web hosting service such as Netlify or Vercel.
Deploy the backend server to a cloud platform like AWS or Heroku.
Ensure proper configuration and security measures are in place for both frontend and backend deployment.\\
\\
\textbf{User Testing and Feedback:}
Invite users to test the platform and provide feedback on usability, performance, and overall experience.
Iterate on the platform based on user feedback and address any issues or improvements identified.\\

\textbf{Maintenance and Updates:}
Monitor the platform for bugs, security vulnerabilities, and performance issues.
Regularly update smart contracts, frontend, and backend components to incorporate new features and improvements.
Stay informed about changes in blockchain technology, third-party dependencies, and regulatory requirements to ensure ongoing compliance and relevance of the platform.

\section{Summary}
This chapter outlines the development process for a blockchain-integrated crowdfunding platform. It starts with setting up the project directory and installing necessary dependencies like Web3.js. Smart contract development in Solidity language includes functions for campaign management and milestone handling, compiled using the Solidity compiler. Frontend development focuses on user interfaces with Web3.js and MetaMask integration. Backend development involves creating API endpoints for interaction with the Ethereum blockchain. Deployment, testing, user feedback, and maintenance ensure a robust and user-friendly platform compliant with regulatory requirements. Results obtained from the project work will be discussed in next chapter. 


\chapter{Results}
\vspace{1cm}

The project reflects a comprehensive implementation of key features aimed at enhancing user experience and functionality. In Fig 5.1 the successful integration of an interactive sidebar, facilitated by React, to streamline navigation within the application is shown. React Router is utilized for seamless page transitions, although this feature is currently non-functional. Navigation links, dynamically sourced from constants, ensure modularity and ease of maintenance, contributing to the platform's scalability. The responsive navbar optimizes user accessibility, offering Metamask wallet connectivity and campaign creation options across different screen sizes. Additionally, an intuitive input form enhances user engagement by simplifying the process of submitting campaign details. These implementations align with the project's objectives of creating a user-friendly and efficient blockchain-based crowdfunding platform. List of a campaign

\begin{figure} [h]
    \centering
    \includegraphics[height=0.3\textheight]{createC.png}
    \caption{{Deployment of the dApp}}
    
\end{figure}

Creating a campaign on the blockchain-based crowdfunding platform involves several key steps. First, the user navigates to the "Create Campaign" section, either through the sidebar or navbar, depending on the platform's layout. Upon clicking the designated button, an input form is displayed, prompting the user to enter essential details about the campaign. These details typically include the campaign title, description, funding goal, duration, and any additional information deemed necessary. Once the user submits the form, the platform utilizes smart contracts deployed on the blockchain to create a new campaign instance. The smart contract records the campaign details, including the creator's address, funding goal, and deadline, ensuring transparency and immutability. Upon successful creation, the platform notifies the user of the campaign's establishment and provides them with a unique identifier or link to share with potential backers. This streamlined process simplifies campaign creation while leveraging blockchain technology to ensure security, transparency, and trust in the crowdfunding process. List of the campaigns that are listed on the applications is shown in Fig 5.2.

\begin{figure} [h]
    \centering
    \includegraphics[scale=0.6]{listofcam.png}
    \caption{{List of a Campaign}}
    
\end{figure}

\textbf{The Functions:}\\
\textbf{createCampaign }- This function likely creates a new crowdfunding campaign on the blockchain.\\
\textbf{getCampaigns} - This function retrieves a list of existing campaigns.\\
\textbf{getDonators} - This function retrieves a list of donors for a specific campaign.\\
\textbf{numberOfCampaigns }- This function retrieves the number of campaigns that have been created.\\
\textbf{Variables:}\\
\textbf{contracts} - This variable stores the addresses of the deployed contracts.\\
\textbf{One specific contract address is shown}: 0x245160150194041616BFD2C8650189794f3c35BA\\
Overall, the code snippet provides a glimpse into the functionalities involved in managing crowdfunding campaigns on the Sepolia network.\\

\begin{figure} [h]
    \centering
    \includegraphics[height=0.3\textheight]{FundB.png}
    \caption{{Dashboard of the dApp}}
    
\end{figure}

The crowdfunding platform interface as shown in 5.3 comprises several key components aimed at facilitating user interaction and campaign management. At the top left corner, users can utilize the search bar to find specific campaigns by entering relevant keywords. However the search for "sampa" yielded no results. Prominently featured on the page is the "Create a campaign" button, which directs users to a dedicated page where they can initiate their crowdfunding campaigns. Beneath this, the "All Campaigns" section showcases a list of active campaigns, with nine examples . Each campaign entry includes essential details such as the campaign title, category (e.g., "Education"), creator name (not displayed ), amount raised (currently all at "0.0"), target goal, progress bar illustrating fundraising progress, days left until the campaign deadline, and a unique wallet address on the  blockchain for direct donations. Notably, the platform operates using Ethereum cryptocurrency for funding, and while the displayed campaigns show no donations, users have the option to support campaigns directly with cryptocurrency via sepolia. Overall, the platform appears to be in its nascent stages of development, with the interface poised to evolve as more campaigns are initiated and funded.\\

\begin{figure} [h]
    \centering
    \includegraphics[height=0.3\textheight]{123.jpg}
    \caption{{Campaigns of Own Accounts}}
    
\end{figure}
In the Accounts section, users can access their accounts and can see the campaigns that are created by them and in campaign details section, users can access crucial information about each project seeking funding as shown in 5.4 . Firstly, the "Campaign title" provides the name of the project, offering insight into its purpose or focus. Following this, the "Campaign description" offers a brief explanation of the project's objectives and how the raised funds will be utilized, providing potential donors with clarity on the initiative's goals. The "Funding goal" indicates the total monetary target that the campaign aims to achieve, giving donors an understanding of the scale of financial support required. Conversely, the "Current amount raised" reveals the accumulated sum of donations received thus far, providing transparency regarding the campaign's progress. The "Time remaining" component specifies the duration left for the campaign, prompting potential donors to act swiftly to contribute before the fundraising period concludes. Additionally, some campaigns may offer "Rewards" to donors as an incentive for their contributions, such as exclusive products or services related to the project. Finally, "Donor recognition" may be included, acknowledging the generosity of contributors by listing their names or showcasing the total number of donors, fostering a sense of community and gratitude within the campaign ecosystem. Overall, these details collectively empower potential donors to make informed decisions about supporting projects aligned with their interests and values.


Funding a campaign involves the process of contributing monetary support to a particular project or initiative through a crowdfunding platform. Users typically navigate to the campaign they wish to support as shown in Fig 5.5 and select a funding option, which may involve various payment methods such as credit/debit cards, bank transfers, or cryptocurrency transactions, depending on the platform's capabilities. Upon selecting the desired amount to contribute, donors proceed to complete the payment process, often guided by intuitive interfaces and secure payment gateways integrated into the crowdfunding platform. Once the payment is successfully processed, the donated amount is added to the campaign's total funds raised, thus inching the project closer to its funding goal. Donors may also have the option to leave comments or messages of encouragement along with their contributions, fostering engagement and community interaction within the crowdfunding ecosystem. Overall, the funding process is designed to be accessible, transparent, and user-friendly, encouraging individuals to support projects they are passionate about and contributing to the realization of impactful initiatives.

\section{Summary}

The project exemplifies a meticulous integration of user-centric features, notably including an interactive sidebar and responsive navigation, which augment user experience by simplifying platform traversal. Campaign creation, a cornerstone functionality, entails users inputting crucial project details, thereby initiating smart contracts on the blockchain to ensure transparency and reliability. The platform interface boasts a search bar for specific campaign queries, a conspicuous "Create a campaign" button, and a comprehensive list of ongoing campaigns, showcasing vital information such as funding progress and remaining duration. Moreover, users gain access to their accounts, facilitating campaign management and project oversight. Funding a campaign involves a streamlined process where users select preferred payment methods and conclude transactions, fostering accessibility and instilling confidence in contributors. Next chapter concludes the report.






