\chapter{Literature Survey}
\vspace{0.5cm}

This chapter provides a comprehensive review of existing research and studies related to the project's topic. It explores various scholarly works, analysis, and findings to establish a foundational understanding.

\section{Related Work}
\\

In the paper [1] titled "Crowdfunding Using Blockchain Technology" by Zad, Saniya, Khan, Zishan, Warambhe, Tejas, Jadhav, Rushikesh, and Alone, Vinod explores the intersection of blockchain technology and crowdfunding mechanisms. The authors investigate how blockchain, known for its decentralized and transparent nature, can enhance traditional crowdfunding models by addressing issues like fraud, intermediary costs, and inefficiencies. By leveraging smart contracts, blockchain can automate funding processes, ensuring that funds are released only when specific conditions are met, thus increasing trust among backers and project creators. The paper highlights various blockchain platforms that have already implemented crowdfunding solutions, offering case studies and comparative analyses. Furthermore, it discusses potential challenges, such as regulatory hurdles and technological limitations, that need to be addressed to fully realize blockchain's potential in crowdfunding.\\

The paper [2] "Public Fund Care Tracking System based on Blockchain" by S. Kumari, T. Dixit, P. Prakash, and V. Sharma, presented at the 2022 2nd Asian Conference on Innovation in Technology (ASIANCON), investigates the application of blockchain technology in tracking public funds. The paper delves into the inefficiencies and lack of transparency in traditional public fund management systems, highlighting issues such as corruption, mismanagement, and the difficulty in tracking the flow of funds. The authors propose a blockchain-based solution to create a transparent, immutable ledger for public fund transactions, ensuring accountability and traceability. By using smart contracts, the proposed system can automate fund disbursement and reporting processes, reducing human error and the potential for fraud. The study includes a detailed discussion of the architecture and implementation of the blockchain-based tracking system, along with its advantages over conventional methods. Additionally, the paper addresses potential challenges, such as scalability, data privacy, and the need for regulatory frameworks to support blockchain integration. Overall, this literature survey provides a thorough analysis of how blockchain technology can improve the management and tracking of public funds, offering a more secure and transparent alternative to existing systems.
\\

The paper presents a comprehensive exploration of blockchain's application in crowdfunding Likely structured as a research or academic article, it offers insights into the potential of decentralized crowdfunding platforms facilitated by blockchain technology. The authors [3] published in IEEE Transactions on Computational Social Systems in 2023, explores the utilization of blockchain technology to enhance the efficiency and transparency of charity processes during crises. The authors examine the traditional challenges faced by charitable organizations, such as delayed fund distribution, lack of transparency, and potential misuse of funds. They propose a blockchain-based system to address these issues, ensuring secure, transparent, and traceable transactions. By leveraging smart contracts, the system can automate the distribution of funds, ensuring they are allocated according to predefined conditions and requirements, thereby minimizing delays and errors. The study provides a detailed analysis of the proposed system's architecture, highlighting its ability to improve trust and accountability among donors and recipients. Additionally, the paper discusses real-world applications and case studies where blockchain has been successfully implemented in charity processes during emergencies, demonstrating its effectiveness. The authors also identify potential challenges, including scalability issues, the need for technological infrastructure, and regulatory concerns, suggesting that these need to be addressed for widespread adoption.

 \\

The paper in [5] titled "How Blockchain Is Revolutionizing Crowdfunding" by Ahmed Banafa, Himanshu Mehra, and Chirag Bansal, published in OpenMind BBVA on July 6, 2020, examines the transformative impact of blockchain technology on crowdfunding. The authors explore how traditional crowdfunding platforms face challenges such as fraud, high intermediary costs, and lack of transparency. Blockchain technology addresses these issues by offering a decentralized and transparent ledger system, which enhances trust and reduces the need for intermediaries. Through the use of smart contracts, blockchain can automate the process of fund distribution, ensuring that funds are released only when predefined conditions are met, thereby increasing accountability and reducing the risk of fraud.

The survey highlights various blockchain-based crowdfunding platforms that have implemented these technologies successfully, providing case studies and comparative analyses to demonstrate their efficacy. Additionally, the authors discuss the potential of tokenization, where backers receive digital tokens representing their contribution, which can be traded or used within the platform’s ecosystem. This not only incentivizes participation but also adds liquidity to the crowdfunding process.
\\

The paper [6]. titled Decentralized Crowdfunding Using Blockchain by Arjun Menon, Kaustubh Kadam, Pranav Kumar, and Subash Kumar Shah, published on January 15, 2023, delves into the innovative application of blockchain technology in the realm of crowdfunding. The authors discuss the limitations of traditional crowdfunding platforms, such as the high reliance on intermediaries, lack of transparency, and vulnerability to fraud. Blockchain technology offers a decentralized solution that addresses these issues by providing a transparent, immutable ledger for transactions.

The survey explains how smart contracts can be utilized to automate various aspects of the crowdfunding process. These self-executing contracts with predefined rules and conditions ensure that funds are only released when specific milestones are achieved, thereby increasing trust between project creators and backers. The paper highlights several decentralized crowdfunding platforms that have successfully adopted blockchain, presenting case studies to showcase the benefits of increased security, reduced costs, and enhanced accountability.
\\
 \section{Summary}

The literature reviewed thus far collectively delves into the extensive impact of blockchain technology on crowdfunding. These papers cover a wide array of topics, including the motivations behind integrating blockchain, the challenges faced, practical implementations, and the transformative effects of blockchain on the crowdfunding ecosystem. In the forthcoming chapter, we will provide an overview of the system architecture and outline the necessary requirements for implementing such a system.