\chapter{Literature Survey}

\vspace{0.5cm}

This chapter provides a comprehensive review of existing research and studies related to multimodal financial forecasting. It explores various scholarly works, technical methodologies, and findings to establish a foundational understanding and motivate the proposed approach.

\section{Time Series Forecasting Models}
\subsection{ARIMA and Classical Methods}
Sheikh Mohammad Idrees et al. (2019) present the ARIMA model for stock market volatility prediction, emphasizing its role as a baseline for benchmarking more advanced models. ARIMA is effective for stationary time series but struggles with non-linear trends and external shocks. Kumar Prakhar et al. (2022) compare ARIMA, ETS, and Prophet, highlighting Prophet's flexibility in handling seasonality and outliers.

\subsection{Prophet Model}
Janmejay Pant et al. (2024) validate the Prophet model on NSE data, demonstrating its modularity and interpretability. Prophet excels in high-frequency and seasonal financial time series, managing outliers and special events with ease. Yash Kumar & Ayan Sikari (2025) benchmark Prophet against ARIMA and LSTM, affirming its advantages in explainability and adaptability for market signal generation.

\section{Sentiment Analysis in Financial Forecasting}
\subsection{VADER and FinBERT}
Qianyi Xiao & Baha Ihnaini (2023) utilize VADER and FinBERT for sentiment analysis, integrating news and social media signals into stock trend prediction. Their work shows that advanced sentiment models significantly improve forecast accuracy, especially when combined with machine learning classifiers.

\subsection{Hybrid Sentiment-Driven Models}
Yingzhe Dong et al. (2020) propose a pipeline using BERT for sentiment extraction and LSTM for price modeling, demonstrating the value of multi-stage NLP integration. Yujia Hu et al. (2021) fuse investor and news sentiment with XGBoost, showing that multi-source sentiment features substantially boost prediction accuracy for tourism sector stocks.

\section{Hybrid and Ensemble Approaches}
\subsection{Motif-Based and Deep Learning Models}
Min Wen et al. (2019) introduce motif-based time series reconstruction combined with CNNs, improving prediction accuracy over ARIMA and LSTM by denoising and capturing complex patterns. Barani Shaju & Valliammal Narayan (2023) present a hybrid LSTM-Neural Prophet model, demonstrating that ensemble architectures reduce error and enhance robustness against outliers and volatility.

\subsection{Sentiment-Enriched Machine Learning}
Gaurav J. Sawale & Manoj K. Rawat (2022) compare sentiment analysis integrated with ML models (SVM, ANN, RF), establishing the effectiveness of multi-input pipelines in capturing market direction and improving trading results.

\section{Critical Comparison and Insights}
The literature collectively demonstrates that integrating advanced time series models (Prophet, motif-based CNN) with sentiment analysis and engineered features leads to improved prediction accuracy and robustness for financial forecasting. Hybrid approaches combining statistical, deep learning, and sentiment-driven signals consistently outperform traditional single-model methods, especially in volatile and noisy market conditions. Feature engineering (motifs, rolling statistics) and news sentiment extraction (via VADER, XGBoost) provide valuable signals that enhance market trend prediction. Benchmarking against classical models (ARIMA, naive, moving averages) confirms that these modern, multimodal architectures deliver more reliable and actionable results for traders, analysts, and financial institutions.
