\chapter{Introduction}

\vspace{0.5cm}

\section{Background and Motivation}
Financial markets are complex systems influenced by a multitude of factors, including stock price movements, macroeconomic indicators, news sentiment, and unpredictable real-world events. Traditional forecasting approaches often rely on a single data modality, such as historical price trends, which can result in incomplete analysis and suboptimal predictions. In recent years, the integration of artificial intelligence (AI) and machine learning (ML) has revolutionized financial forecasting, enabling the fusion of diverse data sources for more robust and accurate market intelligence.

\subsection{Multimodal Fusion in Financial Forecasting}
Multimodal fusion refers to the process of combining information from multiple data types—numerical, textual, and visual—to enhance predictive modeling. In the context of financial forecasting, this involves integrating time series data (e.g., stock prices), sentiment analysis from news and social media, and engineered features such as rolling statistics. By leveraging the strengths of each modality, multimodal systems can capture complex market dynamics and provide actionable insights for traders, analysts, and decision-makers.

\subsection{Project Motivation}
The motivation for this project stems from the limitations of existing financial forecasting systems, which often fail to account for the interplay between market sentiment, price trends, and external events. The goal is to develop a modular AI system that fuses time series analysis, sentiment extraction, and advanced feature engineering to deliver more comprehensive and timely market predictions.

\section{Problem Statement}
Despite advances in financial modeling, most systems still rely on single-source data, such as historical prices or basic technical indicators. This approach overlooks valuable information embedded in news sentiment, social media trends, and short-term market fluctuations. There is a pressing need for an intelligent system capable of integrating and analyzing heterogeneous data sources to generate reliable trading signals and forecasts.

\section{Objectives}
The primary objectives of this project are:
\begin{itemize}
	\item To integrate time series forecasting (using Prophet) and sentiment analysis (using VADER) for enhanced financial prediction.
	\item To engineer rolling statistical features that capture short-term market patterns and anomalies.
	\item To employ XGBoost for robust signal generation, classification, and model benchmarking.
	\item To compare the proposed system against traditional forecasting methods (naive, moving average, linear trend) for objective validation.
	\item To deliver accurate, interpretable, and actionable insights for financial professionals and decision-makers.
\end{itemize}

\section{Report Organization}
This report is organized as follows:
\begin{itemize}
	\item \textbf{Chapter 1: Introduction} — Presents the background, motivation, problem statement, and objectives.
	\item \textbf{Chapter 2: Literature Survey} — Reviews existing research and methodologies in multimodal financial forecasting.
	\item \textbf{Chapter 3: System Architecture and Technical Description} — Details the design, workflow, and implementation of the proposed system.
	\item \textbf{Chapter 4: Results and Evaluation} — Summarizes experimental findings, performance metrics, and visualizations.
	\item \textbf{Chapter 5: Conclusion and Future Scope} — Discusses the contributions, limitations, and potential extensions of the project.
	\item \textbf{References and Appendix} — Provides bibliographic details and supplementary material.
\end{itemize}
